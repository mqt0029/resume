% Document type set to article, for now
% There are prettier templates out there, but let's try this
\documentclass[11pt]{article}

% Beautify packages
\usepackage[utf8]{inputenc}
\usepackage[english]{babel}
\usepackage[T1]{fontenc}
\usepackage{xcolor}
% \usepackage{fontspec}
\usepackage{fontawesome}
\usepackage[sc,bf]{titlesec}
\usepackage{enumitem}
\usepackage{parskip}
\usepackage{tabularx}

% Hyperref tagging
\usepackage[pdftex, 
            pdfauthor={Minh Tram},
            pdftitle={Technical Resume},
            pdfsubject={Resume},
            pdfkeywords={software, engineer, software engineer, software developer, computer
            science, c, c++, java, python, LaTeX, Linux, robotics, computer vision, deep learning,
            NVIDIA, CUDA, OCRTOC, ARIAC, neural network, computer graphics, AI, machine learning, 
            ARIAC, ROS, Docker},
            pdfcreator={VSCode LaTeX Workshop pdfTeX}]{hyperref}

% Page setup
\usepackage[letterpaper,margin=0.5in]{geometry}
\pagenumbering{gobble}

% Placeholder text
\usepackage{lipsum}

\renewcommand*{\faicon}[1]{\makebox[1.5em][c]{\csname faicon@#1\endcsname}}

\begin{document}
% ---------------------------------- Header ---------------------------------- %
    \begin{center}
        \begin{minipage}[b]{0.5\textwidth}
            \raggedright
            {\fontfamily{lmr}\fontsize{35}{35}\selectfont \textsc{Minh Tram}} \\
            \large \faicon{globe} U.S. Permanent Resident 
        \end{minipage}%
        \begin{minipage}[t][0in][b]{0.5\textwidth}
            \raggedleft \small
            linkedin.com/in/mqt0029 \faicon{linkedin-square}\\
            www.mqt0029.github.io \faicon{github} \\
            minh.tram@mavs.uta.edu \faicon{envelope} \\
            (817) 313-7802 \faicon{phone}
        \end{minipage}
    \end{center}
% ---------------------------------------------------------------------------- %
    \section*{Education}
    \hrule height 0.5pt width \textwidth depth 0pt \relax
    \textbf{Master's of Science in Computer Science \hfill Expected May 2022} \\
    % \textbf{Master's of Science in Computer Science \hfill Expected May 2022} \\
    Coursework: Artificial Intelligence, Neural Networks, Machine Learning \hfill Current GPA: 4.00 \\
    Research Thesis: Interactive AR to Robot Interface \\ 
    University of Texas at Arlington - Arlington, TX 

    \textbf{Bachelor of Science in Computer Science \hfill Aug 2017 - June 2020}\\
    Coursework: Algorithms, Operating Systems, Artificial Intelligence, Robotics \hfill GPA: 3.58 \\
    University of Texas at Arlington - Arlington, TX
% ---------------------------------------------------------------------------- %
    \section*{Projects}
    \hrule height 0.5pt width \textwidth depth 0pt \relax
    \vspace{0.1cm}
    \subsection*{IROS 2020 OCRTOC}
    \vspace{-0.2cm}
    % \textbf{IROS 2020 OCRTOC} \\
    \textit{Open Cloud Robot Table Organization Challenge} \vspace {-0.1cm}
    \begin{itemize}[noitemsep]
        \item Utilized NVIDIA DOPE to detect physical objects for robot arm detection and organization tasks
        \item Eliminated hardware integration by simulating hardware characteristics using ROS and
        Gazebo 9 simulator
        \item Created unified development environment to eliminate software discrepancies by creating a
        Docker image
    \end{itemize}
    \subsection*{Real-Time VR-2-Robot Head}
    \vspace{-0.2cm}
    % \textbf{Real-Time VR-2-Robot Head} \\
    \textit{Interactive human-to-robot vision}
    \vspace{-0.1cm}
    \begin{itemize}[noitemsep]
        \item Designed 3-axis head motion simulator to drive mounted stereo camera for user's head motion tracking
        \item Reduced hardware-software lag time by 40\% by packing sensor data through Serial-to-USB asynchronously
        \item Implemented vision and sensor based homing sequence to synchronize robot and human pilot head
        \item Utilized ZED, Unity, and Oculus SDK to stream live video feed for real-time pilot headset visualization
    \end{itemize} 
    \subsection*{Mini Linux}
    \vspace{-0.2cm}
    % \textbf{Mini Linux} \\
    \textit{Minimal Working Example UNIX simulator}
    \vspace{-0.1cm}
    \begin{itemize}[noitemsep]
        \item Implemented bash-like CLI for command handling to provide an effective, interactive command interface
        \item Incorporated persistent FAT32 file system for permanent data storage through file system virtualization
        \item Added multithreading functionality to allow parallel and background software execution by
        using fork()
        \item Ensured safe thread handling to prevent resource hog and deadlocks by utilizing
        semaphores and scheduling
    \end{itemize}
% ---------------------------------------------------------------------------- %
    \section*{Experience}
    \hrule height 0.5pt width \textwidth depth 0pt \relax
    \vspace{0.1cm}
    \subsection*{Graduate Research Assistant \hfill Aug 2020 - Present} 
    \vspace{-0.2cm}
    % \textbf{Graduate Research Assistant \hfill Aug 2020 - Present} \\
    \textit{Active research in robotics and interactive AR pose estimation fusion}
    \vspace{-0.1cm}
    \begin{itemize}[noitemsep]
        \item Implemented OpenCV, PyTorch, Intel RealSense for objects detection and pose estimation on Jetson Nano
        \item Reduced model training time by porting development environment to higher-end machines
        using Docker
        \item Created synthetic training data for object pose detection using Unreal Engine 4 and
        NVIDIA's NDDS
        \item Trained object detection model from 3D stereolithography files for IROS 2020 OCRTOC competition 
    \end{itemize}
% ---------------------------------------------------------------------------- %
    \section*{Technical Skills}
    \hrule height 0.5pt width \textwidth depth 0pt \relax
    \vspace{0.1cm}
    \textbf{Proficient:} Python, ROS, CUDA, \LaTeX, git, Docker $\Vert$ \textbf{Familiar:}
    C\texttt{++}, Java, PyTorch, Tensorflow, Unity\\
    \textbf{Hardware:} NVIDIA Jetson, Universal Robot UR5e, Oculus VR, Intel RealSense, Zed
    Stereo Camera

\end{document}